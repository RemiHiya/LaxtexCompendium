\documentclass[french,titlepage] {article}
%twocolumn
%12pt
\usepackage [] {babel}
\usepackage[T1]{fontenc}
\usepackage[utf8]{inputenc}
\usepackage[]{blindtext}

%maths
\usepackage{mathtools}
\usepackage{amssymb}
\usepackage{amsthm}
%maths


%href
%%%\usepackage[]{hyperref}
%href

%theoremes
\newtheorem {theorem}{Théorème}
%theoremes

%creer des commandes + font
\usepackage[]{lmodern}


\newcommand{\strong}[1]{\mathversion{bold}{\textbf{#1}}}

\newcommand{\ecole}{\Huge U \huge n \LARGE i \Large v \large e \normalsize r \small s \footnotesize i \scriptsize t \tiny é
\Huge C \huge l \LARGE e \Large r \large m \normalsize o \small n \scriptsize t \tiny - \Huge A \huge u \LARGE v \Large e \large r \normalsize g \small n \footnotesize e \scriptsize !}

\newcommand{\texta}[1]{\tiny{#1}}
\newcommand{\textb}[1]{\scriptsize{#1}}
\newcommand{\textc}[1]{\footnotesize{#1}}
\newcommand{\textd}[1]{\small{#1}}
\newcommand{\texte}[1]{\normalsize{#1}}
\newcommand{\textf}[1]{\large{#1}}
\newcommand{\textg}[1]{\Large{#1}}
\newcommand{\texth}[1]{\LARGE{#1}}
\newcommand{\texti}[1]{\huge{#1}}
\newcommand{\textj}[1]{\Huge{#1}}
%creer des commandes



%liens colorées
\usepackage[]{xcolor}
\usepackage[]{soul}
\usepackage[colorlinks=true,urlcolor=blue]{hyperref}
\newcommand{\surligne}[2][pink]{\sethlcolor{#1} \hl{#2}}
\newcommand {\myname} {Cyprien Jullien \xspace}
\newcommand{\mystyle}[2][]{{#1#2}}
\newcommand{\name}[2]{\textcolor{gray}{#1}\textcolor{red}{#2}}
%liens colorées


%lstset

\usepackage[]{listings}
\usepackage[]{graphicx}

\lstset{language=BASH,frame=single, basicstyle=\sffamily,morekeywords=nano,tabsize=2,morekeywords=status,morekeywords=commit,morekeywords=touch,morekeywords=add,morekeywords=clone,morekeywords=push,gobble=0,columns=fixed]
}
%lstset


\title{bienvenue dans le giga résumé}
\author{ Cyprien JULLIEN}
\date{décembre 2023}

\begin{document} %début de l'environement document


\maketitle 
\setcounter{tocdepth}{6}
\tableofcontents

\section{découverte}
language est:
\languagename

titlepage permet de creer une page entiere dédier au titre
twocolumn permet de séparer son texte en 2 sur une page
\blindtext 


\section{maths}

\eqref{equation}

Un peu de maths
 
$3-2+4 \geq 2$

2{10}=1024

$a1,a2\ldots an$

$2^{10} \leq 1024 \times 2 \times 3 \cdots \times 4$


$— les quantificateurs : \exists et \forall ;
— les relations \neq, \equiv, \iff 13 \sim, et \hookrightarrow ;
— les fonctions et opérateurs \cos, \sin, \tan et \lim ;
— le constructeur de fraction \frac qui prend deux arguments et que nous
avons vu en cours-td ;
— des constantes, comme \infty.$


bonus package:
$\nleq, \ngeq et \implies$ 

\begin{equation}
\forall k \leq n, k \textbf{est un diviseur de} n!
\end{equation}

\begin{equation}
e = mc^2 \label{equation}
\end{equation}

\begin{equation}
\lim_{\theta \to (\frac{\pi}{2} ) }
\tan ( \theta ) = \infty
\end{equation}

\subsection{dernier exo}
\Large Que signifient les formules suivantes ?
\begin{enumerate}
\normalsize 
\item $\forall x \in P, \exists y \in P, x \hookrightarrow y$
\item $\forall x \in P, \exists y \in P, y \hookrightarrow x$
\item $\exists x \in P, \exists y \in P, x \hookrightarrow y$
\item $\exists x \in P, \forall y \in P, x \hookrightarrow y$
\item $\exists x \in P, x \hookrightarrow y$

\Large Comment exprimer les assertions suivantes ?
\normalsize 
\item Toute personne admet un père.
\item Certaines personnes sont pères de plusieurs autres.
\item Aucune personne n’a deux pères.
\item Personne n’est le père de son père.
\item La négation de l’assertion 6.
\end{enumerate}


Réponses:
\Large Que signifient les formules suivantes ?
\begin{enumerate}
\normalsize 
\item Toutes personne admet un fils 
\item Toutes personne admet un père 
\item Certaines personnes admet un père
\item Tout fils à un père
\item Il existe des père

\Large Comment exprimer les assertions suivantes ?
\normalsize 
\item $\forall x \in P, \exists y \in P, y \hookrightarrow x$
\item $\exists x \in P, \exists y \in P,\exists z \in P, x \hookrightarrow y,x \hookrightarrow z$
\item $\forall y \in P, \forall x \in P, x \hookrightarrow y$
\item $\forall x \in P, \exists y \in P, x \hookrightarrow y$
\item $\forall x \in P, \exists y \in P, y non \hookrightarrow x$

\end{enumerate}

\section{href et sections}

Mais nous verrons ceci plus tard dans 
\ref{sectionettiquete}
\subsubsection{sous-sous sec}
\blindtext

Mais nous verrons ceci plus tard dans \ref{sectionettiquete}

\subsection{sous sec}

\label{sectionettiquete}
\url{https://fr.wikipedia.org/wiki/Lorem_ipsum}
avec le lien 

\href{https://fr.wikipedia.org/wiki/Lorem_ipsum}{ou en clickant ici}


\section{itemize}

\label{ref2}
\ref{ref2}
\begin{itemize}
\item test
\item test
\item test
\end{itemize}
\begin{enumerate}
\item test
\item test
\item test
\end{enumerate}
\begin{description}
\item test
\item{super} test
\item test
\end{description}



\section{equations}
$ \mathbb{N} = [1,2,3,...]$

\begin{theorem}[Pour tout entier naturel n, la somme des n premiers entiers
naturels est égal à la moitié du produit de n par $(n  -  1)$.]

\begin{proof}
Soit $ n\in\mathbb { N } $ .
Notons $ S_n $ la somme des $ n $ premiers entiers naturels .
Alors :
\begin{align*}
S_n & = \sum_ { i = 0 } ^ { n - 1 } i
= 0 + 1 + \cdots + ( n - 2) + ( n - 1)
& \text { pardé finition } \\
& = \sum_ { i = 0 } ^ { n - 1 } ( n - 1 - i )
= ( n - 1) + ( n - 2) + \cdots + 1 + 0
& \text { par commutativit é } \\
\end{align*}
En sommant les deux formes de la somme $ S_n $ donn é es ci - dessus,
nous pouvons obtenir :
\label{1ererep}
\begin{align*}
2 \times S_n
& = \left ( \sum_ { i = 0 } ^ { n - 1 } i\right )
+ \left ( \sum_ { i = 0 } ^ { n - 1 } ( n - 1 - i ) \right ) \\
& = \sum_ { i = 0 } ^ { n - 1 } ( i + ( n - 1 - i ))
& \text { par associativit é et commutativit é } \\
& = Sn = \frac {n \times(n - 1)}{2} 
\end{align*}
\emph{ Réponse  }
$Sn = \frac {n \times(n - 1)}{2} $
\end{proof}




\end{theorem}

\begin{theorem}(version textuelle).Pour tout entier naturel n, la somme des n premiers entiers
naturels est égal à la moitié du produit de n par $(n  -  1)$.]
\begin{equation}
S_n  = \sum_ { i = 0 } ^ { n - 1 } i
= 0 + 1 + \cdots + ( n - 2) + ( n - 1)
 \text { pardé finition } \\
 = \sum_ { i = 0 } ^ { n - 1 } ( n - 1 - i )
= ( n - 1) + ( n - 2) + \cdots + 1 + 0
 \text { par commutativit é } \\
\end{equation}

\begin{equation*}
2 \times S_n
 = \left ( \sum_ { i = 0 } ^ { n - 1 } i\right )
+ \left ( \sum_ { i = 0 } ^ { n - 1 } ( n - 1 - i ) \right ) \\
 = \sum_ { i = 0 } ^ { n - 1 } ( i + ( n - 1 - i ))
 \text { par associativit é et commutativit é } \\
 = Sn = \frac {n \times(n - 1)}{2} 
\end{equation*}

Initialisation: Cas n = 0 \\
Hérédité:\\
On sait que Pn est vrai, c'est a dire...\\
Montrons que Pn+1 est vrai, c'est a dire..\\
Conclusion:\\
Donc d'après le principe de récurrence Pn+1 est vrai pour tout i appartenant à R



\end{theorem}

\begin{theorem}(version textuelle).Pour tout entier naturel n, la somme des n premiers entiers
\begin{equation}
S_n  = \sum_ { i = 0 } ^ { n - 1 } i
= 0 + 1 + \cdots + ( n - 2) + ( n - 1)
 \text { pardé finition } \\
 = \sum_ { i = 0 } ^ { n - 1 } ( n - 1 - i )
= ( n - 1) + ( n - 2) + \cdots + 1 + 0
 \text { par commutativit é } \\
\end{equation}

\begin{equation*}
2 \times S_n
 = \left ( \sum_ { i = 0 } ^ { n - 1 } i\right )
+ \left ( \sum_ { i = 0 } ^ { n - 1 } ( n - 1 - i ) \right ) \\
 = \sum_ { i = 0 } ^ { n - 1 } ( i + ( n - 1 - i ))
 \text { par associativit é et commutativit é } \\
 = Sn = \frac {n \times(n - 1)}{2} 
\end{equation*}
ou la retrouver ici --> \ref{1ererep}
\end{theorem}

\section{tableaux}

\begin{tabular}{|l|l|}
\hline
  \textbf{quantificateur} & \textbf{signification} \\
  \hline
  \hline
   $\forall$ & Pour tout    \\
   \hline
  $\exists$ & Il existe     \\
   \hline
\end{tabular}
\\
\centering{Table 1 – Tableau de guillemets} \\
\begin{tabular}{cccccc}
\textbf{nom} & \textbf{anglais} & \textbf{utf8} & \textbf{code} \LaTeX & \textbf{rendu} \\
\hline
 simples guillemets & \textit{simple quotes} & ‘hey’ &
\verb{`{ hey \verb{'{  & ‘hey’\\
 doubles guillemets & \textit{double quotes} & “hey” &
\verb{``{ hey \verb{''{ & “hey”\\
 guillemets français & \textit{french quotes} & «hey» & \verb{ \og{ hey \verb{\fg{ & « hey » \\
\end{tabular}

\section{github}

\centering{Table 1 – Tableau de GIT} \\
\begin{tabular}{|l|c|}
\hline
\textbf{commande} & \textbf{utilité} \\
\hline
\hline
 git add "fichier" & permet d'ajouter le fichier dans le git\\
 git commit -m "message" & creer une version et ajouter un message\\
 git status & affiche l'état de la version\\
 git log & affiche la liste des versions\\
 git init & demande a git de suivre le repertoire\\
 git diff [fichier] & affiche les modification d'un fichier\\
 rm -fr .git & supprimer le git\\
 git config --global user.name/email  & modifier le nom de l'utilisateur\\
 git commit --amend --reset-author --no-edit & appliquer les modif ci dessus\\
 git push & envoyer l'update à git lab\\
 git pull & recevoir l'update de git lab\\
 git clone "url" & clone le git lab sur notre git\\
 
 \hline
\end{tabular}

 on peut aussi creer un fichier .gitinit avec un *.aux pour ne pas prendre en compte les fichier .aux 
 
\section{fontsize + creer des commandes custom}
 
\fontsize{56}{1}\selectfont
\textbf
Je suis en police par défaut.\texttt{Je suis en police sans empattement.}\textsf{Je suis en police à
chasse fixe.} \textrm{Je suis en police roman, explicitement.}
\fontsize{1}{1}\selectfont
\begin{itemize}

\item 1. Je ne change la police par défaut. . .
\ttfamily
\item 2. . . . qu’à partir de maintenant, pour une police sans empattement. . .
\item 3. . . . qui continue un petit moment. . .
\item 4. . . . avant d’être finalement changée. . 
\rmfamily
\item 5. . . . pour la police par défaut. . .
\sffamily
\item 6. . . . puis pour la police à chasse fixe...
\item 7. ... qui continue...
\item 8. ... encore et encore, mais pas jusqu’au bout.
\rmfamily
\item Après la liste, je retrouve la police par défaut.
\end{itemize}

\ecole

\texta tiny
\textb scriptsize \textc footnotesize
\textd small \texte normalsize
\textf large \textg Large
\texth LARGE \texti huge
\textj Huge \normalsize – voilà mes différentes tailles.

\strong{$ 1+1$}

\textbf{
$ \mathversion{bold} Attention, 2 + 2 = 2 \times 2mais3 + 3 \ne 3 \times 3.$}


\textit{L’italique est différent du slanted.} 
\textsl {Le slanted est différent du texte droit.}
 \textup{Le texte
droit est très clairement différent du texte en petites capitales.}

\section{creer des liens colorées}

\href{appelle}
\ref{appelle}

\sethlcolor{green}
\hl{ h1}

\surligne{le texte est en vert}

\surligne{}{le texte est en de couleur de base}


\begin{enumerate}



\item  \mystyle[\bfseries\slshape]{maman - papa}
\item \mystyle[\bfseries\slshape]{ \myname - frère - soeur}


\end{enumerate}

\textcolor{blue}{bleu} \textcolor{green}{vert} \textcolor{red}{rouge} \textcolor{gray}{gris}
Ça, \mystyle[\bfseries\slshape]{c’est} \name{Cyprien}{Jullien} et il a du \mystyle[\scshape\large]{style} !


\section[test]{lstisting + flottant et verbatim}


\begin{enumerate}

\item cloner! \ref{image1} petit 1\begin{lstlisting}
git clone url depoB
\end{lstlisting} 
\label{image1}

\item creer fichier tex
\ref{image2}petit 2

\item \begin{lstlisting}
git touch correction.tex
git commit -m "ajout d'un fichier"
git push
\end{lstlisting} 
\label{image2}

\item creer fichier .gitignore
\ref{image3}petit 3
\begin{lstlisting}

touch .gitignore
nano .gitignore

\end{lstlisting} 
 \label{image3}
%% recherche d'une image dans le meme dossier \lstinputlisting[frame=single]{.gitignore}

\item \item git status \ref{image4}petit 4
\begin{lstlisting}
git status
\end{lstlisting} 
\label{image4}
\begin{figure}[h]
%\includegraphics[scale=0.5]{cap.png}

\item 

\ref{image}image
\caption{ceci est le git status \label{image}}
\end{figure}
\begin{lstlisting}
git reponse suivante


\end{lstlisting}
\end{enumerate}



\end{document}